\documentclass{article}
\usepackage{geometry}
\usepackage{graphicx} % Required for inserting images
\usepackage{amsmath, amsthm, amssymb}
\usepackage{parskip}
\newgeometry{vmargin={15mm}, hmargin={24mm,34mm}}
\theoremstyle{definition} 
\newtheorem{definition}{Definition}

\newtheorem{theorem}{Theorem}[section]
\newtheorem{lemma}[theorem]{Lemma}
\newtheorem{corollary}{Corollary}[theorem]
\newtheorem{example}{Example}[theorem]

\newcommand{\N}{\mathbb{N}}
\newcommand{\Z}{\mathbb{Z}}
\newcommand{\C}{\mathbb{C}}
\newcommand{\R}{\mathbb{R}}
\newcommand{\Q}{\mathbb{Q}}
\newcommand{\fdiff}{f^{\prime}}
\newcommand{\interior}{\text{int}}
\newcommand{\dom}{\text{Dom}}
\newcommand{\ran}{\text{Ran}}

\title{Complex Analysis}
\date{March 2024}

\begin{document}

\maketitle

\section{Resources}

\begin{itemize}
    \item Stein and Shakarchi
    \item Serge Lang (I didn't like how he didn't introduce topology. It's lame.)
    \item BrightSide of Math Youtube Series
\end{itemize}

\section{Introduction to Complex Numbers}

\begin{example}
    Consider $f: \C \xrightarrow{} \C$ defined by $f(z) = \frac{1}{z}$. This function maps the unit circle outside of the unit circle
    and maps the outside of the unit circle to the unit circle. It's a bijection. It's called an \textbf{inversion} through the unit circle. 
\end{example}

\begin{example}
    Consider $f: \C \xrightarrow{} \C$ defined by $f(z) = \frac{1}{\bar{z}}$. This function maps the unit circle outside of the unit circle
    and maps the outside of the unit circle to the unit circle. It's a bijection. It's called an \textbf{reflection} through the unit circle. 
\end{example}

\begin{lemma}
    Let $z_{n}$ be a sequence of complex numbers with $z_{n} = x_{n} + i y_{n}$. $z_{n}$ is Cauchy if and only if
    $x_{n}$ and $y_{n}$ are Cauchy. 
\end{lemma}
\begin{proof}
    The forward direction is immediate since $\lvert x_{n} - x_{m} \rvert \leq \lvert z_{n} - z_{m} \rvert$ and similarly
    for $y_{n}$.

    Notice that $\lvert z_{n} - z_{m} \rvert = \sqrt{(x_{n} - x_{m})^{2} + (y_{n} - y_{m})^{2}}$, which proves the converse.
\end{proof}

\begin{definition}
    Let $f: U \xrightarrow{} \C$. $f$ is \textbf{(complex) differentiable} at $z_{0} \in U$ there's $f^{\prime}(z_{0}) \in \C$
    and $\phi: \C xrightarrow{} \C$ with

    \[ f(z) = f(z_{0}) + f^{\prime}(z_{0})(z - z_{0})  + \phi(z)\]

    where $\lim_{z \to z_{0}} \frac{\phi(z)}{z - z_{0}} = 0$.
\end{definition}

How do we prove the uniqueness of the derivative?

\begin{definition}
    Let $f: U \xrightarrow{} \C$. $f$ is \textbf{holomorphic} if $f$ is complex differentiable on $U$.
\end{definition}

\begin{definition}
    Let $f_{R}: \R^{2} \xrightarrow{} \R^{2}$. $f$ is \textbf{totally differentiable}

\end{definition}



\end{document}