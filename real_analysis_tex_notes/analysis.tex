\documentclass{article}
\usepackage{geometry}
\usepackage{graphicx} % Required for inserting images
\usepackage{amsmath, amsthm, amssymb}
\usepackage{parskip}
\newgeometry{vmargin={15mm}, hmargin={24mm,34mm}}
\theoremstyle{definition} 
\newtheorem{definition}{Definition}

\newtheorem{theorem}{Theorem}[section]
\newtheorem{lemma}[theorem]{Lemma}
\newtheorem{corollary}{Corollary}[theorem]

\newcommand{\N}{\mathbb{N}}
\newcommand{\Z}{\mathbb{Z}}
\newcommand{\R}{\mathbb{R}}
\newcommand{\Q}{\mathbb{Q}}
\newcommand{\fdiff}{f^{\prime}}
\newcommand{\interior}{\text{int}}
\newcommand{\dom}{\text{Dom}}
\newcommand{\ran}{\text{Ran}}

\title{Analysis Notes}
\date{March 2024}

\begin{document}

\maketitle

Can you have 3 disjoint dense sets in a set?

\section{Metric Spaces}

\begin{lemma}
    Let $(X,\rho)$ be a metric space and $A \subseteq X$. $g: X \xrightarrow{} \R$ defined by
    
    \[ g(x) = \rho(x,A)\]

    is a continuous function.
\end{lemma}
\begin{proof}
    We'll prove a stronger condition, namely, we'll prove that $\forall x,y \in X: \lvert g(x) - g(y) \rvert \leq \rho(x,y)$.

    Let $x,y \in X$. Observe that for any $p \in A$,
    
    \[g(x) = \inf_{a \in A} \rho(x,a) \leq \rho(x,y) + \rho(y,p) \]

    Then, $g(x) - \rho(x,y)$ is a lower bound on $A = \{\rho(y,a): a \in A\}$ and thus
    $g(x) - \rho(x,y) \leq g(y)$ by the \textit{greatest} lower bound property of the
    infimum. Rearranging produces the desired inequality.
\end{proof}

In fact, notice that it is uniformly continuous.

\begin{lemma}
    Let $(X,\rho)$ be a metric space and $A \subseteq B \subseteq X$.
    Then, $diamA \leq diamB$.
\end{lemma}

\newpage

\section{Metric Space Compactness}

\begin{lemma}
    Let $(X,\rho)$ be a complete metric space. Let $E_{n}$ be a sequence of non-empty,
    closed and bounded subsets such that $\forall n \in \N: E_{n+1} \subseteq E_{n}$.
    If $\lim_{n \to \infty} diam(E_{n}) = 0$, $\bigcap_{i = 1}^{\infty} E_{n}$
    has exactly one point.

    In other words, each nested sequence of shrinking closed
    sets in a complete space has exactly one point in common.
\end{lemma}
\begin{proof}
    Before we prove the (harder) existence, let's first prove uniqueness.
    Assume by contradiction that $E$ contains two distinct points $p$
    and $q$. Let $d := \frac{1}{2} \rho(p,q)$. Then, $diam(E) > d$.
    However, $p$ and $q$ are also in every $E_{n}$ so we also just
    got a lower bound for $diam(E_{n})$ for all $n \in \N$, which contradicts
    the fact that the diameters are going to 0.

    Let's now prove that the intersection is not empty. Since $E$
    is an intersection of closed sets, $E$ is closed. Thus, it suffices
    to find a limit point of $E$.

    By invoking the Axiom of Choice, pick some $x_{n} \in E_{n}$ for all
    $n \in \N$. We'll prove that this sequence is Cauchy. 
    
    Let $\epsilon > 0$.
    Since $\lim_{n \to \infty} diam(E_{n}) = 0$, there's some $N \in \N: \forall n \geq N:
    diam(E_{n}) < \epsilon$. Then, by using the fact that the sequences are nested, we immediately
    get that $x_{n}$ is 
\end{proof}

\newpage

\section{Infinite Series}

\subsection{Cauchy Condensation Test}

\begin{theorem}
    Let $a_{n}$ be a non-negative and decreasing sequence. Then,

    \[ \sum_{n=1}^{\infty} a_{n}\leq 2^{n} \sum_{n = 1}^{\infty} a_{2^{n}}\leq 2 \sum_{n=1}^{\infty} a_{n} \]
\end{theorem}


\subsection{Rearrangement Theorems}

\begin{theorem}
    Any rearrangement of an absolutely convergent series converges to the same sum.
\end{theorem}

The idea behind the proof is to use the Cauchy Criterion and cancel out the
first $N$ terms by taking $m$ to be large enough. The rest of the proof is just
fun and games with the triangle inequality.

\subsection{Dirichlet and Abel Summation Tests}

\section{Continuity}

\begin{lemma}
    The inverse of a monotonic function is monotonic in the same way.
\end{lemma}
\begin{proof}
    Without loss of generality, assume $f$ is monotonically increasing.
    Then, $x < y \implies f(x) < f(y)$ so $f^{-1}(x) < f^{-1}(y) \implies x < y$.
    Considering the contrapositive gives the required result.
\end{proof}

\newpage

\section{Intermediate Value Theorem}

\begin{theorem}[Intermediate Value Theorem]
    
\end{theorem}

\begin{corollary}
    Every odd-degree polynomial has a root in $\R$.
\end{corollary}

\begin{corollary}\label{fixed_point_ivt}
    Let $I \subseteq \R$ be a compact interval and $f:I \xrightarrow{} I$ be a continuous function 
    with $\dom(f) = I$. Then, $\exists x \in I: f(x) = x$.
\end{corollary}
\begin{proof}
    
\end{proof}


\section{Differentiation}

\subsubsection{Inverse Function Rule}

The following lemma is from Exercise 5.2 in Rudin:

\begin{lemma}
    Let $f: [a,b] \xrightarrow{} \R$ and assume
    $\fdiff(x) > 0$ on $(a,b)$. Let $g$ be the inverse of $f$.
    Then, $g$ is differentiable and
    
    \[ g^{\prime}(f(x)) = \frac{1}{\fdiff(x)} \]
\end{lemma}
\begin{proof}

\end{proof}

Here's a homework exercise from MATH131BH.
Notice that we need $f$ to be injective for the
inverse to be defined. Also notice that no regularity
assumptions are made about $f$ apart from the
behavior at $x$. In particular, $f$ might be discontinuous
and monotone on $(x - \delta, x + \delta)$ for any $\delta > 0$.

\begin{lemma}
    Let $f: \R \xrightarrow{} \R$ be an injective function on $\dom(f)$. Let $x \in \interior(\dom(f))$
    such that $f(x) \in \interior(\ran(f))$. Assume that 
    the inverse function is continuous at $f(x)$.
    If $\fdiff(x) \neq 0$, $f^{-1}$ is differentiable at $f(x)$
    with

    \[ (f^{-1})^{\prime}(x) = \frac{1}{\fdiff(x)} \]
\end{lemma}
\begin{proof}
    
\end{proof}

The requirement that $f^{-1}$ is continous can be replaced
by different assumptions. For example, $f$ can be continously differentiable.

Ross has another version of this lemma.

\newpage

\section{Applications of Uniform Convergence: }

\newpage

\section{Fixed Point Theorems}

A good resource is Raymond Chu's notes for MATH204 at UCLA.

The first of the fixed point theorems was a corollary to IVT: see Corollary \ref{fixed_point_ivt}.

\begin{theorem}[Banach Contraction Mapping Theorem]
    Let $(X,d)$ be a non-empty complete metric space. Let $f: X \xrightarrow{} X$
    be $c$-Lipschitz for some $c \in (0,1)$. Then, $f$ has a unique fixed point.
\end{theorem}
\begin{proof}
    The uniqueness is immediate by the Lipschitz condition.

    We'll consider the sequence given by continuously applying $f$. Since $X$ is complete,
    it suffices to show that this sequence is Cauchy.
\end{proof}

We can't take $c = 1$ since for example $f(x) = x + 1$ satisfies it but has no fixed points.

\newpage

\section{Banach Spaces}

\begin{definition}
    Let $X$ be an $F$ vector space and $d$ be a metric induced by a norm on $X$. Assume 
    $(X,d)$ is a complete metric space. This is called a \textbf{Banach space}.
\end{definition}




\end{document}